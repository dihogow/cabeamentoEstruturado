\begin{table}[h!]
	\onehalfspacing
	\caption{Tabela Explicativa dos Elementos Constituintes da Rede}
	\vspace{0.5cm}
	\centering
	\renewcommand{\arraystretch}{1.4}
	\label{tab3}
	\resizebox{\textwidth}{!}{%
	\begin{tabular}{|l|l|}
		\hline
		\multicolumn{1}{|c|}{\textbf{Elemento}} & \multicolumn{1}{c|}{{\color[HTML]{000000} \textbf{Função}}}                                                                                                                                                                                                        \\ \hline
		DGT                                     & {\color[HTML]{000000} Distribuidor Geral de Telecomunicações}                                                                                                                                                                                                      \\ \hline
		{\color[HTML]{000000} SEQ}              & {\color[HTML]{000000} Sala de Equipamentos}                                                                                                                                                                                                                        \\ \hline
		A, B, C, D, E, F, G                     & {\color[HTML]{000000} Cabeamento Horizontal Sobre Forro}                                                                                                                                                                                                           \\ \hline
		ART                                     & {\color[HTML]{000000} \begin{tabular}[c]{@{}l@{}}Cabeamento Horizontal Baixo, encaminhado via calhas de rodapé, os \\ quais atendem cada uma das áreas de trabalho. Sua identificação é composta \\ por: ART+”Nº da área de trabalho correspondente”\end{tabular}} \\ \hline
		PTXN                                    & {\color[HTML]{000000} \begin{tabular}[c]{@{}l@{}}Pontos de Rede RJ45 com sua devida identificação, composta pelo padrão: \\ PT+”Letra do Cabeamento Horizontal Sobre Forro”+”Nº do Ponto”\end{tabular}}                                                            \\ \hline
	\end{tabular}
}
\end{table}